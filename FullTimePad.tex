\documentclass[a4paper,12pt]{article}

\usepackage{graphicx}
\usepackage{geometry}
\usepackage{listings}
\usepackage[hidelinks]{hyperref}
\usepackage[svgnames, table]{xcolor}

\newcommand{\subsubsubsection}[1]{\paragraph{#1}\mbox{}\\}

\title {
		\Huge \textbf{Full-Time-Pad\\Symmetric Stream Cipher} \\
	\ \\
	\ \\
	\ \\
	\ \\
	\ \\
	\Large \textbf{Improved One-Time-Pad Encryption Scheme}

}

\author{Taha Canturk\\\texttt{kibnakanoto@protonmail.com}}
\date{2024-05-20}


\begin{document}
\maketitle
\thispagestyle{empty}

\pagenumbering{roman}

\begin{center}
		\Large \texttt{Version 1.0}
		\ \\
		\ \\
		\ \\
		\ \\
		\small License to copy this document is granted provided it is identified as "Full-Time-Pad", in all material mentioning or referencing it.
\end{center}

\newpage


\begin{abstract}
	Abstract
\end{abstract}

\newpage

\tableofcontents

\newpage

\pagenumbering{arabic}

\section{Introduction}



\subsection {Pre-requisite Terminology}


\subsection{Applications}


\subsection{Key Generation}


\subsection{Prerequisite Mathematics}


\subsection{Vector Permutation}


\section{Security Vulnerabilities}



\subsection{Brute-Force}


\subsubsection {Birthday Attack}

\subsubsection {Denial of Service (DoS)}

\subsection{Reverse Engineering the Transformation}

\subsection{Collision-Resistance}

\subsubsection{Different Permutation Matrices}

\subsubsection{Number of Rounds}

\subsubsection{Constant - $F_p$ - Prime Galois Field Size} % Fp and r

\subsubsection{Constant - $r$ - Dynamic Rotation Constant} % Fp and r

\section{Hashing}



\subsection{Diffusion - Permutation}


\subsubsection{Vector Permutation}


\subsubsection{Dynamic vs. Static}

\subsection{Dynamic Matrix Permutation}


% mention python code

\subsubsection{Deravation}

\subsubsection{Other Options} % include Discovery

\subsection{Confusion - ARX}

\subsubsection{A - Modular Addition} % pseudo-code and Explanation

\subsubsection{R - Bitwise Rotation} % pseudo-code and Explanation

\subsubsection{X - XOR} % pseudo-code and Explanation

\subsection{Key Transformation} % pseudo-code and Explanation - brings it all together

\section {Cipher}

\subsection{Transformation}

\subsection {Avalanche Effect - Plaintext}

\subsubsection {Encryption Index}

\subsection {Long Plaintexts}

\end{document}

