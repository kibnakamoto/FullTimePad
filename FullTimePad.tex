\documentclass[fleqn, a4paper,12pt]{article}

\usepackage{graphicx}
\usepackage{geometry}
\usepackage{listings}
\usepackage[hidelinks]{hyperref}
\usepackage[svgnames, table]{xcolor}
\usepackage{algorithm} % pseudo code
\usepackage{mathtools} % equations
\usepackage{amssymb} % equations
\usepackage{algpseudocode}  % Correctly load this for pseudocode
\usepackage{tikz} % drawing trailing lines in pseudo code
\usepackage[makeroom]{cancel}

\setlength{\arraycolsep}{2pt} % Adjust the horizontal spacing between elements

\setcounter{MaxMatrixCols}{32}

\geometry{
  left=1in,    % Left margin
  right=1in,   % Right margin
  top=1in,     % Top margin
  bottom=1in   % Bottom margin
}

\title {
	\Huge \textbf{Full-Time-Pad\\Symmetric Stream Cipher} \\
	\ \\
	\ \\
	\ \\
	\ \\
	\ \\
	\Large \textbf{Improved One-Time-Pad Encryption Scheme}

}

\author{Taha Canturk\\\texttt{kibnakanoto@protonmail.com}}
\date{2024-05-20}


\begin{document}
\maketitle
\thispagestyle{empty}

\pagenumbering{roman}

\begin{center}
		\Large \texttt{Version 1.0}
		\ \\
		\ \\
		\ \\
		\ \\
		\small License to copy this document is granted provided it is identified as "Full-Time-Pad", in all material mentioning or referencing it.
\end{center}

\newpage


\begin{abstract}
		\fontsize{12}{18}\selectfont \texttt{One-Time-Pad} Encrypion Scheme is a secure algorithm but there are 2 main security risks. One, a key cannot be reused. Two, plaintext length equals key length which is very inefficient when dealing with long plaintexts. These 2 security risks only exist due to a lack of confusion and diffusion per ciphertext. As denoted by Claude Shannon in the report he published in 1945, A Mathematical Theory of Cryptography, A secure cryptographic algorithm requires confusion and diffusion. The \texttt{Full-Time-Pad} symmetric stream cipher is developed based on the \texttt{One-Time-Pad} with solutions to the security risks while maintaining high speed computation. \
		To achieve diffusion, the key is permutated in it's byte array form using a constant permutation matrix. To acheive the confusion, the key is manipulated in it's 32-bit integer representation using Modular \textbf{A}ddition in $F_p$, Bitwise \textbf{R}otations, and \textbf{X}or (\textbf{ARX}). The permutation guarantees that every time there is a manipulation, eacj 32-bit number is made up of a different byte order.

		% TODO: how is the uneiqeness of each key gaurenteed, incrementing key, IV. etc.
		% TODO: how to deal with long messsages
		% TODO: This paragraph is for introduction, abstract should be talking about what's in the document

\end{abstract}

\newpage

\tableofcontents

\newpage

\pagenumbering{arabic}

\section{Introduction}



\subsection {Pre-requisite Terminology}
 
\renewcommand{\arraystretch}{2} % Increases vertical space between rows 

\begin{tabular}{l p{12cm}}  % left-aligned 2 columns
		\textbf{Key}              & \hangafter=1 \hangindent=1.19cm \hspace{1cm} 32-byte random array that's transformed, then hashed before XORed with the plaintext to encrypt \\
		\textbf{Symmetric}        & \hangafter=1 \hangindent=1.19cm \hspace{1cm} Same key is used for encryption and decryption \\
		\textbf{Stream}           & \hangafter=1 \hangindent=1.19cm \hspace{1cm} Plaintext is encrypted without seperating it into blocks \\
		\textbf{Plaintext}        & \hangafter=1 \hangindent=1.19cm \hspace{1cm} Plain data before encryption \\
		\textbf{Ciphertext}       & \hangafter=1 \hangindent=1.19cm \hspace{1cm} Encrypted plaintext \\
		\textbf{Cipher}           & \hangafter=1 \hangindent=1.19cm \hspace{1cm} Encryption algorithm. Plaintext is transformed into a ciphertext that can only be reversed with a key \\
		\textbf{Diffusion}        & \hangafter=1 \hangindent=1.19cm \hspace{1cm} plaintext/key is spread out in the ciphertext \\
		\textbf{Confusion}        & \hangafter=1 \hangindent=1.19cm \hspace{1cm} The ciphertext has no possible statistical analysis, or cryptoanalysis to determine the plaintext \\
		\textbf{Bit}              & \hangafter=1 \hangindent=1.19cm \hspace{1cm} 0 or 1. Smallest discrete unit for computation \\
		\textbf{Byte}             & \hangafter=1 \hangindent=1.19cm \hspace{1cm} 8-bit number \\
		\textbf{Galois Field}     & \hangafter=1 \hangindent=1.19cm \hspace{1cm} Finite Field where there are only limited number of numbers. Only prime galois fields ($F_p$) are used where size of the field is denoted by prime number p \\
		\textbf{Avalanche Effect} & \hangafter=1 \hangindent=1.19cm \hspace{1cm} An aspect of diffusion. If smallest unit (1 bit) of data is changed, the ciphertext changes in an unrecognizable way.  \\
\end{tabular}



\subsection{Applications}


\subsection{Key Generation}

The 32-byte key should be generated using a cryptographically secure method, including but not limited to cryptographic random number generators and Elliptic Cryptography Diffie Hellman (ECDH) protocol with Hash-based Key Derivation Function (HKDF)

\subsection{Prerequisite Mathematics}


\subsection{Vector Permutation}


\section{Security Vulnerabilities}
\setlength{\mathindent}{3pt} % No indentation for equations

In One-Time-Pad, key isn't reusable. Here is the proof:

\[
\begin{aligned}
&\texttt{let }  m_1, m_2 \texttt{ be 2 plaintexts} \\
&\texttt{let }  k \texttt{ be the key} \\
&\texttt{let }  c_1 = m_1 \oplus k \\
&\texttt{let }  c_2 = m_2 \oplus k \\
&c_1 \oplus c_2 = (m_1 \oplus k) \oplus (m_2 \oplus k) \\
&c_1 \oplus c_2 = m_1 \oplus m_2
\end{aligned}
\]

Since the key is reused, the 2 ciphertext's XORed factor out the key since $k \oplus k = 0$
Using cryptoanalysis, the 2 plaintexts can be found. 
\\
For $c_1 \oplus c_2 = m_1 \oplus m_2$ to not hold true, for each encryption, the key needs to be different. If $k$ is transformed each time so that it has an avalanche effect. Even with no confusion, it would still be secure since $k' \oplus k \neq  0$ where $k'$ is transformed key. 
\\
But there is another concern,
\\
What if the plaintext and ciphertext are known, then it is possible to find $k$ so don't use k without transformation, since $\texttt{plaintext} \oplus \texttt{ciphertext} = \texttt{key}$. So for each plaintext, key needs to be transformed irreversibly and it also requires confusion since if $k'$ is found, $k$ is still unknown but if $k$ is found, then all instances of $k'_n$ are known, which means that: 
\[
\begin{aligned}
&k'_1 = hash(k+1) \texttt{ where hash() is an irreversible transformation} \\
&k'_2 = hash(k+2) \\
&c_1 \oplus c_2 = (m_1 \oplus k'_1) \oplus (m_2 \oplus k'_2) \\
&c_1 \oplus c_2 \neq m_1 \oplus m_2 \\
&m_1 \oplus c_1 = k'_1 \\
&m_2 \oplus c_2 = k'_2 \\
&k'_1, k'_2 \texttt{  are calculated using an irreversible hashing algorithm }
\end{aligned}
\]
\\


$\therefore$ the \texttt{Full-Time-Pad} Cipher requires both diffusion and confusion

\subsection{Brute-Force}

Due to the use of a galois field. The total number of combinations per 256-bit key isn't $a=2^{256}$, but rather $b=4294967291^8$ where $p=4294967291$ for arithmetic in $F_p$ and there are 8 32-bit numbers in a 256-bit key.

\[
\begin{aligned}
&a = 115792089237316195423570985008687907853269984665640564039457584007913129639936_{10} \\
&b = 115792088158918333131516597762172392628570465465856793992332884130307292657121_{10} \\
&\texttt{let  } \Delta = a-b \\
&\Delta = 1078397862292054387246515515224699519199783770047124699877605836982815_{10} \\
\end{aligned}
\]
So the difference $\Delta$ is a somewhat large integer. The number of combinations with a galois field is lower than without a galois field ($b < a$). This isn't a big concern as their difference measured exponentially is only around $2^{\log_2\Delta} \approx 2^{229}$ which means that their difference is around $2^{229}$, this is a negligible difference as the difference between $2^{230}$ and $2^{229}$ is also huge. \\
$\therefore $ \texttt{Using a galois field doesn't negatively impact number of cominations in terms of brute force as the total number of combinations when using a galois field vs not is a negligible amount}

\subsubsection {Birthday Problem} \label{birthday_problem}

The birthday problem is a paradox. It goes as follows: how many people are required so that there is more than 50\% chance that at least 2 people have the same birthday.
\\
The answer is an unexpected 23 people.
\\
In the context of this encryption algorithm, it might be a concern, as number of key reused (with transformation) increase, the chances of finding the key increase:
\\
\\

\texttt{let } $V_c$ \texttt{be the number of combinations per key without order and repetitions} \\
\texttt{let } $k$ \texttt{be the number of keys needed for hash(key) to have a 50\% chance to equal another hash(key)} \\
\texttt{let } $V_t$ \texttt{be the number of combinations per key with order and repetitions} \\

\[
\begin{aligned}
&V_c = \frac{b!}{(b-k)!} = \frac{4294967291^8!}{(4294967291^8-k)!} \\
&V_t = b^k = 4294967291^{8^k} \\
&P(A) = \frac{V_c}{V_t} \\
&P(A) = \frac{\frac{b!}{(b-k)!}}{b^k} \\
&P(B) = 1- P(A) = 50\%\\
&P(A) = 1 - 50\%\\
&1 - 50\% = \frac{\frac{b!}{(b-k)!}}{b^k} \\
&\frac{1}{2}b^k =  \frac{b!}{(b-k)!} \hspace{7.2cm} \texttt{ since }50\% = \frac{1}{2} \\
&\log_b\frac{1}{2}b^k = \log_b\frac{b!}{(b-k)!} \\
&\log_b\frac{1}{2} + \log_bb^k = \log_bb! - \log_b(b-k)! \\
&0 = \log_bb! - \log_b(b-k)! - \log_b\frac{1}{2} - k \hspace{3.2cm} \texttt{ since }\log_bb^k = k\\
\end{aligned}
\]
% seperate equations to 2 segments otherwise previous page is empty
\thispagestyle{empty} % page number is confusing in the equation

\small
\[
\begin{aligned}
& \\
& \texttt{According to Ramanujan's Approximation: } \\
& \quad \log_bb! \approx \frac{b\ln b - b + \frac{\ln\biggl[\frac{1}{\pi^3} + b(1+4b(1+2b)) \biggr]}{6} + \frac{ln\pi}{2}}{\ln b} \\
& \texttt{And} \\
& \\
& \log_b(b-k)! \approx \frac{(b-k)\ln(b-k) - (b-k) + \frac{\ln \biggl[\frac{1}{\pi^3} +  (b-k)(1+4(b-k)(1+2(b-k)))\biggr]}{6} + \frac{\ln\pi}{2}}{\ln b} \\
& \\
& \texttt{Recall: } \\
& \quad 0 = \log_bb! - \log_b(b-k)! - \log_b\frac{1}{2} - k \hspace{2.8cm} \texttt{ isolate } \log_b(b-k)! \\
& \quad \log_b(b-k)! = \log_bb! - \log_b\frac{1}{2} - k \\
& \\
& \texttt{Combine both equations for } \log_b(b-k)! \texttt{: } \\
& \log_bb! - \log_b\frac{1}{2} - k \approx \frac{(b-k)\ln(b-k) - (b-k) + \frac{\ln \biggl[\frac{1}{\pi^3} + (b-k)(1+4(b-k)(1+2(b-k)))\biggr]}{6} + \frac{\ln\pi}{2}}{\ln b} \\
& \\
& \frac{b\ln b - b + \frac{\ln\biggl[\frac{1}{\pi^3} +  b(1+4b(1+2b)) \biggr]}{6} + \frac{\ln\pi}{2}}{\ln b} - \log_b\frac{1}{2} - k \approx \frac{(b-k)\ln(b-k) - (b-k)}{\ln b} +  \\ 
& \hspace{9cm} + \frac{\frac{\ln \biggl[\frac{1}{\pi^3} +  (b-k)(1+4(b-k)(1+2(b-k))) \biggr]}{6} + \frac{\ln\pi}{2}}{\ln b} \\
\end{aligned}
\]

\thispagestyle{empty} % page number is confusing in the equation
\[
\begin{aligned}
		& \frac{b\log b \cancel{-b} + \frac{\ln\biggl[\frac{1}{\pi^3} +  b(1+4b(1+2b)) \biggr]}{6} + \cancel{\frac{ln\pi}{2}} - \ln b \log_b\frac{1}{2} - \ln b k}{\cancel{\ln b}} \approx \frac{(b-k)\ln(b-k) \cancel{-b} + k }{\cancel{\ln b}} +  \\ 
& \hspace{10.3cm} + \frac{\frac{\ln \biggl[\frac{1}{\pi^3} + (b-k)(1+4(b-k)(1+2(b-k)))\biggr]}{6} + \cancel{\frac{\ln\pi}{2}}}{\cancel{\ln b}} \\
& \\
& \texttt{let } C = b\ln b + \frac{\ln \biggl[\frac{1}{\pi^3} + b(1+4b(1+2b)) \biggr]}{6} - \ln b \log_b\frac{1}{2} \approx (b-k)\ln(b-k) + k + \ln b k  \\ 
& \hspace{10.1cm} + \frac{\ln \biggl[\frac{1}{\pi^3} + (b-k)(1+4(b-k)(1+2(b-k)))\biggr]}{6} \\
& \\
& \texttt{let } f(k) = (b-k)\ln(b-k) + k + \ln b k + \frac{\ln \biggl[\frac{1}{\pi^3} + (b-k)(1+4(b-k)(1+2(b-k)))\biggr]}{6} - C = 0\\
%& C \approx \log(b-k)^{b-k} + k + \log b k + \frac{\log \biggl[ (b-k)(1+4(b-k)(1+2(b-k)))\biggr]}{6} \\
%& C \approx k(1 + \log b) + \frac{\log \biggl[ (b-k)^{6(b-k)+1}(1+4(b-k)(1+2(b-k)))\biggr]}{6} \\
\end{aligned}
\]

\[
\begin{aligned}
%& \texttt{let } x = b-k \\
%& C \approx k(1 + \log b) + \frac{\log \biggl[ x^{6x+1}(1+4x+8x^2) \biggr]}{6} \\
%& 6C \approx 6k(1 + \log b) + \log \biggl[ x^{6x+1} (1+4x+8x^2) \biggr] \\
%& \log \biggl[ x^{6x+1} (1+4x+8x^2) \biggr] \approx 6C - 6k(1 + \log b) \\
%& \log( x^{6x+1} + 4x^{6x+2} + 8x^{6x+3}) \approx 6C - 6k(1 + \log b) \\
%& \texttt{let } f(k) = \log( x^{6x+1} + 4x^{6x+2} + 8x^{6x+3}) - 6C + 6k(1 + \log b) = 0 \\
%& x^{6x+1} + 4x^{6x+2} + 8x^{6x+3} = 10^{6C - 6k(1 + \log b)} \\
%& x^{6x+1} + 4x^{6x+2} + 8x^{6x+3} - 10^{6C - 6k(1 + \log b)} = 0 \\
%& \texttt{let } f(k) = x^{6x+1} + 4x^{6x+2} + 8x^{6x+3} - 10^{6C - 6k(1 + \log b)} = 0 \\
\end{aligned}
\]
$\therefore$ $f(k)$ \texttt{ can be used to evaluate how many keys it would take so that 2 hashes have a 50\% chance of being equal. $f(k)$ can be evaluated using the secant algorithm}

After running \texttt{test/secant.py}, given the parameters: \\
Based on Wikipedia Article: Birthday Attack, we can approximate $x_0$ and $x_1$ \\
$x_0 = \frac{1}{2}+\sqrt{\frac{1}{4}+2 \times ln(2) \times b}$ (due to Approximation of number of people) \\
$x_1 = \sqrt{b}$ (due to square approximation) \\
\\ error tolarance:
$e = 1 \times 10^{-200}$ \\
for $b=4294967291^8$,  we get $k_1 = 400651867432320527534628274526034254879$ for the root. \\
And for $b = 2^{256}$, we get $k_2 = 400651869298001176472314306405665023048$ for the root \\
So then $\Delta k = k_2 - k_1 = 1865680648937686031879630768169 \approx 2^{101}$
Since the difference between $k_1$ and $k_2$ is negligible ($2^{101}$ isn't big considering the magnitude of $b$). We can conclude  that using a galois field doesn't increase risk of birthday attacks which justifies the use of Galois fields to increase avalanche effect.

% TODO: TEST BRUTE FORCING IT TO GET HIGHEST POSSIBLE COLLISION CHANCES FROM ANY 32-BYTES

\subsubsection {Denial of Service (DoS)}

Most denial of service attacks related to encryption algorithms are based on brute-force methods. To see if this algorithm has a potential collision attack: \\
$\texttt{transform(} key_1 \texttt{)} = \texttt{transform(} key_2 \texttt{)}$ \\
\\
For example: $x + y = 16$ \\

$ x,y \in \mathbb{Z}, 0 \leq x,y < 256$ \\
there are 17-combinations for x to satisfy this equation, and simultaniously, there are 17 combinations for y to satisfy the equation, so a total of $17$ combinations.
\\
But for $x + y = z$, there are $257$ combinations to try. if the result of an arithmetic operation is known, there may be ways to get the same end-result with less combinations to brute-force. Knowing the value of $z$ reduced the number of combinations by $15$ times.
\\
This means that the calculation done on \ref{birthday_problem} for the birthday problem would be irrelevant because there is a better algorithm than random brute forcing (to find collisions for \texttt{transform(key)})

$\therefore$ if there is an operation that can provide the same output for a wide range of inputs, there can be a collision attack. Collision attacks can be used to derive the same transformed key using a different input key and decrypt the plaintext without actually having the original key.
\\
In the context of this encryption algorithm (using addition as an example): keysize is 32-bytes
\\
so for byte n: $x_n + y_n = z_n$\\
$ x,y,z \in \mathbb{Z}, 0 \leq x,y,z < 256$\\
\\
Number of combinations can be represented by 
\[
\\		\prod_{n=0}^{32-1}(z_n+1)
\]
\small so the number of combinations would be between a minimum of 32 combinations ($z_n=0$ for all 32-bytes) up to a maximum of $2^{256}$ combinations ($z_n = 255$ for all 32-bytes) which can be brute forced for small $z_n$. So a simple addition is prone to collision attacks for $x_n+y_n=z_n$, where $x,y$ are unknown. The use of galois field makes $z_n$ even smaller. So even less combinations. Solution is to use operations that cannot be represented differently. e.g.
\[
\\\sum_{i=0}^{z_n} x_n + y_n = z_n \Longrightarrow
\sum_{i=0}^{z_n} (z_n-i)+(i) = z_n
\]

solves for all possible $x, y$ values for each $z_n$. An addition operation can be represented differently to solve for 2 unknowns, while a good mix of ARX operations cannot be reverse engineered. This is also the reason why pre-manipulating the key (using addition) before \texttt{transform()} isn't a good option. Since it provides a very obvious collision attack which makes it invalid even though pre-manipulation will provide a good avalanche effect for every single byte of the key (if 1-bit of any byte is changed, ciphertext changes completely).

So the final solution is to calculate sum of each 32-bit segment of the key (represented by $k_i$) in order to interlink them to make sure that every byte of the key offers the same avalanche effect:
\[
\\\sum_{i=0}^{7} k_i
\]

To test if this offers enough collision resistance: think of this problem as an example:
\\
\textbf{1.} $x+y=16 \texttt{\hspace{2.3cm} offers n=17 combinations}$\\
\textbf{2.} $x+y+z=16 \texttt{\hspace{1.7cm} offers n=153 combinations (determined experimentally)}$\\
\textbf{3.} $x+y+z+v=16 \texttt{\hspace{1cm} offers n=969 combinations (determined experimentally)}$\\
$x,y,z,v \in \mathbb{Z}, 0 \leq x,y,z,v < 256$\\

So there has to be an equation or algorithm to summarize the relationship between number of variables ($l$) and the sum of the addition operation ($16$);

Knowing that equation \textbf{1.} is the simplest equation and it offers 17 combinations. Then if the rest of the equations are represented in 2-variable fashion. we can find number of combinations $n$:

For equation \textbf{2.}: There are 3 ways to represent as 2-variable equation
\begin{center}
\[
\hspace*{\fill} x + y \hspace{1cm} x + z \hspace{1cm} y + z \hspace*{\fill}
\]
\end{center}

For equation \textbf{3.}: There are 6 ways to represent as 2-variable equation
\begin{center}
\[
\hspace*{\fill} x + y \hspace{1cm} x + z \hspace{1cm} x + v \hspace*{\fill}
\]
\end{center}
\begin{center}
\[
\hspace*{\fill} y + z \hspace{1cm} y + v \hspace{1cm} z + v \hspace*{\fill}
\]
\end{center}

The number of ways a multi-variable equation can be represented as a 2 variable equation can be summarized by the following:

\[
\\ \sum_{i=1}^{l-1} i
\]

Using some number crunching and logic, I found that there is a direct corrolation between the number of combinations and the ratio between the current number of ways to represent as 2-variable equation over the previous number of ways to represent as a 2-variable equation:

\[
		\\ n_l \propto \frac{\displaystyle \sum_{i=1}^{l-1}}{\displaystyle \sum_{i=1}^{l-2}}
\]

Using more number crunching: I found the following recursive formula that finds the number of combinations that satisfies x+y+\dots = 16:

\[
		\\ n_l = (n_{l-1} \frac{\displaystyle \sum_{i=1}^{x}}{\displaystyle \sum_{i=1}^{x-1}} +17) \times 3 - 17 \times 3( (x+1) \mod 2 )
\]
where $x = l-1$ \quad and x should be incremented until correct answer is reached for $l \geq 6$ and $n_{l-1}$ is previous number of combinations.
This formula doesn't translate to cases where the 2-variable equation doesn't have 17 combinations.

Simply put this equation couldn't be used accurately, it can only be an approximation. But upon further number crunching, I derived the following equation that satisfies all cases:

\[
\\ x \prod_{i=1}^{l-1} \frac{(x+i)}{1+i}
\]
\\
where x is the number of combinations for 2 variable equations. e.g. for $a+b = 16, x = 16+1 = 17$.

Using this equation for the context of this encryption algorithm:

Recall:
\[
\\ c = \sum_{i=1}^{l-1} i
\]
where $l=8$ since 8 32-bit segments to the 256-bit key
then, $x = c + 1$

the total number of combinations according to the equation is between $1$ and \\$2871827628774669857283799072180574717903946432793745331030345747716374528 \approx 2.9 \times 10^{72}$

Which isn't possible to brute force provided key is random and not chosen to be a small value.

But according to testing, summation rarely provide low avalanche effect for certain input keys. But using xor operator offers the same avalanche effect for all keys and it offers $2^{256} \approx 1.2 \times 10^{77}$ (more) combinations.

\subsection{Reverse Engineering the Transformation}

\subsection{Collision-Resistance}

\subsubsection{Different Permutation Matrices}

tried permutation matrices that followed logic or randomness. But they didn't offer the proper diffusion and collision resistance required to make a secure algrorithm. The permutation matrix needs to be perfect so that the chances of collision (tested in \texttt{test/significant\_perm\_byte.cpp}) for every byte of the key should be around the same.

\subsubsection{Number of Rounds}

\subsubsection{Constant - $F_p$ - Prime Galois Field Size} % Fp and r

\subsubsection{Constant - $r$ - Dynamic Rotation Constant} % Fp and r

\section{Hashing}



\subsection{Diffusion - Permutation}

\subsubsection{Dynamic vs. Static}

\subsection{Dynamic Matrix Permutation}


% mention python code

\subsubsection{Deravation}

\begin{algorithm}
\caption{Dynamic Permutation Matrix Deravation Pseudo-code}
\begin{algorithmic}[1]  % The [1] enables line numbering
\State \textbf{Input:} an array of incrementing numbers (0-31) $A$ 
\State \textbf{Output:} Most Efficient Permutation Matrix $V$ ($16 \times 32$)
\State \textbf{Begin}
\State $P \gets \texttt{copy of A}$
\For{$k = 0$ to $4$}
	\For{$i = 0$ to $8$}
		\State $P_i      \gets A_{i \times 4}$
		\State $P_{i+8}  \gets A_{i \times 4 + 1}$
		\State $P_{i+16} \gets A_{i \times 4 + 2}$
		\State $P_{i+24} \gets A_{i \times 4 + 3}$
	\EndFor
	\State $A \gets \texttt{copy of P}$
	\State $V.append(P)$
	\State $C \gets \texttt{copy of P}$
	\For{$m = 0$ to $3$}
		\For{$i = 0$ to $8$}
			\For{$n = 0$ to $4$}
			\State $P_{i \times 4 + n} \gets C_{(1+n+m) \mod{4} \texttt{    } + \texttt{    } i \times 4}$
			\EndFor
		\EndFor
		\State $V.append(P)$
	\EndFor
	\State $A \gets \texttt{copy of P}$
\EndFor
\State \textbf{Return} $V$
\end{algorithmic}
\end{algorithm}

Python code is in the test/perm.py

 
\subsubsection{Dynamic Permutation Matrix Values} % include Discovery

\renewcommand{\arraystretch}{1} % Decrease vertical space between rows 

\small
\[
\begin{Bmatrix}
\phantom{0}0 & \phantom{0}4 & \phantom{0}8 & 12 & 16 & 20 & 24 & 28 & \phantom{0}1 & \phantom{0}5 & 9 & 13 & 17 & 21 & 25 & 29 & \phantom{0}2 & \phantom{0}6 & 10 & 14 & 18 & 22 & 26 & 30 & \phantom{0}3 & \phantom{0}7 & 11 & 15 & 19 & 23 & 27 & 31 \\
\phantom{0}4 & \phantom{0}8 & 12 & \phantom{0}0 & 20 & 24 & 28 & 16 & \phantom{0}5 & 9 & 13 & \phantom{0}1 & 21 & 25 & 29 & 17 & \phantom{0}6 & 10 & 14 & \phantom{0}2 & 22 & 26 & 30 & 18 & \phantom{0}7 & 11 & 15 & \phantom{0}3 & 23 & 27 & 31 & 19 \\
\phantom{0}8 & 12 & \phantom{0}0 & \phantom{0}4 & 24 & 28 & 16 & 20 & 9 & 13 & \phantom{0}1 & \phantom{0}5 & 25 & 29 & 17 & 21 & 10 & 14 & \phantom{0}2 & \phantom{0}6 & 26 & 30 & 18 & 22 & 11 & 15 & \phantom{0}3 & \phantom{0}7 & 27 & 31 & 19 & 23 \\
12 & \phantom{0}0 & \phantom{0}4 & \phantom{0}8 & 28 & 16 & 20 & 24 & 13 & \phantom{0}1 & \phantom{0}5 & 9 & 29 & 17 & 21 & 25 & 14 & \phantom{0}2 & \phantom{0}6 & 10 & 30 & 18 & 22 & 26 & 15 & \phantom{0}3 & \phantom{0}7 & 11 & 31 & 19 & 23 & 27 \\
12 & 28 & 13 & 29 & 14 & 30 & 15 & 31 & \phantom{0}0 & 16 & \phantom{0}1 & 17 & \phantom{0}2 & 18 & \phantom{0}3 & 19 & \phantom{0}4 & 20 & \phantom{0}5 & 21 & \phantom{0}6 & 22 & \phantom{0}7 & 23 & \phantom{0}8 & 24 & 9 & 25 & 10 & 26 & 11 & 27 \\
28 & 13 & 29 & 12 & 30 & 15 & 31 & 14 & 16 & \phantom{0}1 & 17 & \phantom{0}0 & 18 & \phantom{0}3 & 19 & \phantom{0}2 & 20 & \phantom{0}5 & 21 & \phantom{0}4 & 22 & \phantom{0}7 & 23 & \phantom{0}6 & 24 & 9 & 25 & \phantom{0}8 & 26 & 11 & 27 & 10 \\
13 & 29 & 12 & 28 & 15 & 31 & 14 & 30 & \phantom{0}1 & 17 & \phantom{0}0 & 16 & \phantom{0}3 & 19 & \phantom{0}2 & 18 & \phantom{0}5 & 21 & \phantom{0}4 & 20 & \phantom{0}7 & 23 & \phantom{0}6 & 22 & 9 & 25 & \phantom{0}8 & 24 & 11 & 27 & 10 & 26 \\
29 & 12 & 28 & 13 & 31 & 14 & 30 & 15 & 17 & \phantom{0}0 & 16 & \phantom{0}1 & 19 & \phantom{0}2 & 18 & \phantom{0}3 & 21 & \phantom{0}4 & 20 & \phantom{0}5 & 23 & \phantom{0}6 & 22 & \phantom{0}7 & 25 & \phantom{0}8 & 24 & 9 & 27 & 10 & 26 & 11 \\
29 & 31 & 17 & 19 & 21 & 23 & 25 & 27 & 12 & 14 & \phantom{0}0 & \phantom{0}2 & \phantom{0}4 & \phantom{0}6 & \phantom{0}8 & 10 & 28 & 30 & 16 & 18 & 20 & 22 & 24 & 26 & 13 & 15 & \phantom{0}1 & \phantom{0}3 & \phantom{0}5 & \phantom{0}7 & 9 & 11 \\
31 & 17 & 19 & 29 & 23 & 25 & 27 & 21 & 14 & \phantom{0}0 & \phantom{0}2 & 12 & \phantom{0}6 & \phantom{0}8 & 10 & \phantom{0}4 & 30 & 16 & 18 & 28 & 22 & 24 & 26 & 20 & 15 & \phantom{0}1 & \phantom{0}3 & 13 & \phantom{0}7 & 9 & 11 & \phantom{0}5 \\
17 & 19 & 29 & 31 & 25 & 27 & 21 & 23 & \phantom{0}0 & \phantom{0}2 & 12 & 14 & \phantom{0}8 & 10 & \phantom{0}4 & \phantom{0}6 & 16 & 18 & 28 & 30 & 24 & 26 & 20 & 22 & \phantom{0}1 & \phantom{0}3 & 13 & 15 & 9 & 11 & \phantom{0}5 & \phantom{0}7 \\
19 & 29 & 31 & 17 & 27 & 21 & 23 & 25 & \phantom{0}2 & 12 & 14 & \phantom{0}0 & 10 & \phantom{0}4 & \phantom{0}6 & \phantom{0}8 & 18 & 28 & 30 & 16 & 26 & 20 & 22 & 24 & \phantom{0}3 & 13 & 15 & \phantom{0}1 & 11 & \phantom{0}5 & \phantom{0}7 & 9 \\
19 & 27 & \phantom{0}2 & 10 & 18 & 26 & \phantom{0}3 & 11 & 29 & 21 & 12 & \phantom{0}4 & 28 & 20 & 13 & \phantom{0}5 & 31 & 23 & 14 & \phantom{0}6 & 30 & 22 & 15 & \phantom{0}7 & 17 & 25 & \phantom{0}0 & \phantom{0}8 & 16 & 24 & \phantom{0}1 & 9 \\
27 & \phantom{0}2 & 10 & 19 & 26 & \phantom{0}3 & 11 & 18 & 21 & 12 & \phantom{0}4 & 29 & 20 & 13 & \phantom{0}5 & 28 & 23 & 14 & \phantom{0}6 & 31 & 22 & 15 & \phantom{0}7 & 30 & 25 & \phantom{0}0 & \phantom{0}8 & 17 & 24 & \phantom{0}1 & 9 & 16 \\
\phantom{0}2 & 10 & 19 & 27 & \phantom{0}3 & 11 & 18 & 26 & 12 & \phantom{0}4 & 29 & 21 & 13 & \phantom{0}5 & 28 & 20 & 14 & \phantom{0}6 & 31 & 23 & 15 & \phantom{0}7 & 30 & 22 & \phantom{0}0 & \phantom{0}8 & 17 & 25 & \phantom{0}1 & 9 & 16 & 24 \\
10 & 19 & 27 & \phantom{0}2 & 11 & 18 & 26 & \phantom{0}3 & \phantom{0}4 & 29 & 21 & 12 & \phantom{0}5 & 28 & 20 & 13 & \phantom{0}6 & 31 & 23 & 14 & \phantom{0}7 & 30 & 22 & 15 & \phantom{0}8 & 17 & 25 & \phantom{0}0 & 9 & 16 & 24 & \phantom{0}1 \\
\end{Bmatrix}
\]

\subsubsection{Other Options} % include Discovery

\subsection{Confusion - ARX}

\subsubsection{A - Modular Addition} % pseudo-code and Explanation

\subsubsection{R - Bitwise Rotation} % pseudo-code and Explanation

\subsubsection{X - XOR} % pseudo-code and Explanation

\subsection{Key Transformation} % pseudo-code and Explanation - brings it all together

\section {Cipher}

\subsection{Transformation}

\subsection {Avalanche Effect - Plaintext}

\subsubsection {Encryption Index}

\subsection {Long Plaintexts}

\end{document}

